% Options for packages loaded elsewhere
\PassOptionsToPackage{unicode}{hyperref}
\PassOptionsToPackage{hyphens}{url}
%
\documentclass[
]{article}
\usepackage{amsmath,amssymb}
\usepackage{lmodern}
\usepackage{iftex}
\ifPDFTeX
  \usepackage[T1]{fontenc}
  \usepackage[utf8]{inputenc}
  \usepackage{textcomp} % provide euro and other symbols
\else % if luatex or xetex
  \usepackage{unicode-math}
  \defaultfontfeatures{Scale=MatchLowercase}
  \defaultfontfeatures[\rmfamily]{Ligatures=TeX,Scale=1}
\fi
% Use upquote if available, for straight quotes in verbatim environments
\IfFileExists{upquote.sty}{\usepackage{upquote}}{}
\IfFileExists{microtype.sty}{% use microtype if available
  \usepackage[]{microtype}
  \UseMicrotypeSet[protrusion]{basicmath} % disable protrusion for tt fonts
}{}
\makeatletter
\@ifundefined{KOMAClassName}{% if non-KOMA class
  \IfFileExists{parskip.sty}{%
    \usepackage{parskip}
  }{% else
    \setlength{\parindent}{0pt}
    \setlength{\parskip}{6pt plus 2pt minus 1pt}}
}{% if KOMA class
  \KOMAoptions{parskip=half}}
\makeatother
\usepackage{xcolor}
\usepackage[margin=1in]{geometry}
\usepackage{color}
\usepackage{fancyvrb}
\newcommand{\VerbBar}{|}
\newcommand{\VERB}{\Verb[commandchars=\\\{\}]}
\DefineVerbatimEnvironment{Highlighting}{Verbatim}{commandchars=\\\{\}}
% Add ',fontsize=\small' for more characters per line
\usepackage{framed}
\definecolor{shadecolor}{RGB}{248,248,248}
\newenvironment{Shaded}{\begin{snugshade}}{\end{snugshade}}
\newcommand{\AlertTok}[1]{\textcolor[rgb]{0.94,0.16,0.16}{#1}}
\newcommand{\AnnotationTok}[1]{\textcolor[rgb]{0.56,0.35,0.01}{\textbf{\textit{#1}}}}
\newcommand{\AttributeTok}[1]{\textcolor[rgb]{0.77,0.63,0.00}{#1}}
\newcommand{\BaseNTok}[1]{\textcolor[rgb]{0.00,0.00,0.81}{#1}}
\newcommand{\BuiltInTok}[1]{#1}
\newcommand{\CharTok}[1]{\textcolor[rgb]{0.31,0.60,0.02}{#1}}
\newcommand{\CommentTok}[1]{\textcolor[rgb]{0.56,0.35,0.01}{\textit{#1}}}
\newcommand{\CommentVarTok}[1]{\textcolor[rgb]{0.56,0.35,0.01}{\textbf{\textit{#1}}}}
\newcommand{\ConstantTok}[1]{\textcolor[rgb]{0.00,0.00,0.00}{#1}}
\newcommand{\ControlFlowTok}[1]{\textcolor[rgb]{0.13,0.29,0.53}{\textbf{#1}}}
\newcommand{\DataTypeTok}[1]{\textcolor[rgb]{0.13,0.29,0.53}{#1}}
\newcommand{\DecValTok}[1]{\textcolor[rgb]{0.00,0.00,0.81}{#1}}
\newcommand{\DocumentationTok}[1]{\textcolor[rgb]{0.56,0.35,0.01}{\textbf{\textit{#1}}}}
\newcommand{\ErrorTok}[1]{\textcolor[rgb]{0.64,0.00,0.00}{\textbf{#1}}}
\newcommand{\ExtensionTok}[1]{#1}
\newcommand{\FloatTok}[1]{\textcolor[rgb]{0.00,0.00,0.81}{#1}}
\newcommand{\FunctionTok}[1]{\textcolor[rgb]{0.00,0.00,0.00}{#1}}
\newcommand{\ImportTok}[1]{#1}
\newcommand{\InformationTok}[1]{\textcolor[rgb]{0.56,0.35,0.01}{\textbf{\textit{#1}}}}
\newcommand{\KeywordTok}[1]{\textcolor[rgb]{0.13,0.29,0.53}{\textbf{#1}}}
\newcommand{\NormalTok}[1]{#1}
\newcommand{\OperatorTok}[1]{\textcolor[rgb]{0.81,0.36,0.00}{\textbf{#1}}}
\newcommand{\OtherTok}[1]{\textcolor[rgb]{0.56,0.35,0.01}{#1}}
\newcommand{\PreprocessorTok}[1]{\textcolor[rgb]{0.56,0.35,0.01}{\textit{#1}}}
\newcommand{\RegionMarkerTok}[1]{#1}
\newcommand{\SpecialCharTok}[1]{\textcolor[rgb]{0.00,0.00,0.00}{#1}}
\newcommand{\SpecialStringTok}[1]{\textcolor[rgb]{0.31,0.60,0.02}{#1}}
\newcommand{\StringTok}[1]{\textcolor[rgb]{0.31,0.60,0.02}{#1}}
\newcommand{\VariableTok}[1]{\textcolor[rgb]{0.00,0.00,0.00}{#1}}
\newcommand{\VerbatimStringTok}[1]{\textcolor[rgb]{0.31,0.60,0.02}{#1}}
\newcommand{\WarningTok}[1]{\textcolor[rgb]{0.56,0.35,0.01}{\textbf{\textit{#1}}}}
\usepackage{graphicx}
\makeatletter
\def\maxwidth{\ifdim\Gin@nat@width>\linewidth\linewidth\else\Gin@nat@width\fi}
\def\maxheight{\ifdim\Gin@nat@height>\textheight\textheight\else\Gin@nat@height\fi}
\makeatother
% Scale images if necessary, so that they will not overflow the page
% margins by default, and it is still possible to overwrite the defaults
% using explicit options in \includegraphics[width, height, ...]{}
\setkeys{Gin}{width=\maxwidth,height=\maxheight,keepaspectratio}
% Set default figure placement to htbp
\makeatletter
\def\fps@figure{htbp}
\makeatother
\setlength{\emergencystretch}{3em} % prevent overfull lines
\providecommand{\tightlist}{%
  \setlength{\itemsep}{0pt}\setlength{\parskip}{0pt}}
\setcounter{secnumdepth}{-\maxdimen} % remove section numbering
\ifLuaTeX
  \usepackage{selnolig}  % disable illegal ligatures
\fi
\IfFileExists{bookmark.sty}{\usepackage{bookmark}}{\usepackage{hyperref}}
\IfFileExists{xurl.sty}{\usepackage{xurl}}{} % add URL line breaks if available
\urlstyle{same} % disable monospaced font for URLs
\hypersetup{
  pdftitle={Parsing Vectors and Files},
  pdfauthor={Jarred Robidoux},
  hidelinks,
  pdfcreator={LaTeX via pandoc}}

\title{Parsing Vectors and Files}
\author{Jarred Robidoux}
\date{2023-02-17}

\begin{document}
\maketitle

\hypertarget{parsing-a-vector}{%
\section{\texorpdfstring{\textbf{Parsing a
Vector}}{Parsing a Vector}}\label{parsing-a-vector}}

Let's talk about the \textbf{parse}() functions. These functions take a
character vector and return a more specialized vector like a logical,
integer, or date.

\begin{Shaded}
\begin{Highlighting}[]
\FunctionTok{str}\NormalTok{(}\FunctionTok{parse\_logical}\NormalTok{(}\FunctionTok{c}\NormalTok{(}\StringTok{"TRUE"}\NormalTok{, }\StringTok{"FALSE"}\NormalTok{, }\StringTok{"NA"}\NormalTok{)))}
\end{Highlighting}
\end{Shaded}

\begin{verbatim}
##  logi [1:3] TRUE FALSE NA
\end{verbatim}

\begin{Shaded}
\begin{Highlighting}[]
\FunctionTok{str}\NormalTok{(}\FunctionTok{parse\_integer}\NormalTok{(}\FunctionTok{c}\NormalTok{(}\StringTok{"1"}\NormalTok{, }\StringTok{"2"}\NormalTok{, }\StringTok{"3"}\NormalTok{)))}
\end{Highlighting}
\end{Shaded}

\begin{verbatim}
##  int [1:3] 1 2 3
\end{verbatim}

\begin{Shaded}
\begin{Highlighting}[]
\FunctionTok{str}\NormalTok{(}\FunctionTok{parse\_date}\NormalTok{(}\FunctionTok{c}\NormalTok{(}\StringTok{"2010{-}10{-}01"}\NormalTok{, }\StringTok{"1979{-}10{-}14"}\NormalTok{)))}
\end{Highlighting}
\end{Shaded}

\begin{verbatim}
##  Date[1:2], format: "2010-10-01" "1979-10-14"
\end{verbatim}

Like all function in the tidyverse, the \textbf{parse}() functions are
uniform: the first argument is a character vector to parse, and the na
argument specifies which strings should be treated as missing:

\begin{Shaded}
\begin{Highlighting}[]
\FunctionTok{parse\_integer}\NormalTok{(}\FunctionTok{c}\NormalTok{(}\StringTok{"1"}\NormalTok{, }\StringTok{"231"}\NormalTok{, }\StringTok{"."}\NormalTok{, }\StringTok{"456"}\NormalTok{), }\AttributeTok{na =} \StringTok{"."}\NormalTok{)}
\end{Highlighting}
\end{Shaded}

\begin{verbatim}
## [1]   1 231  NA 456
\end{verbatim}

If parsing fails, you'll get a warning:

\begin{Shaded}
\begin{Highlighting}[]
\NormalTok{x }\OtherTok{\textless{}{-}} \FunctionTok{parse\_integer}\NormalTok{(}\FunctionTok{c}\NormalTok{(}\StringTok{"123"}\NormalTok{, }\StringTok{"345"}\NormalTok{, }\StringTok{"abc"}\NormalTok{, }\StringTok{"123.45"}\NormalTok{))}
\end{Highlighting}
\end{Shaded}

\begin{verbatim}
## Warning: 2 parsing failures.
## row col               expected actual
##   3  -- no trailing characters abc   
##   4  -- no trailing characters 123.45
\end{verbatim}

If there are many parsing failures, you can use \textbf{problems()} to
get the complete set. This returns a tibble, which you can then
manipulate with dplyr

\begin{Shaded}
\begin{Highlighting}[]
\FunctionTok{problems}\NormalTok{(x)}
\end{Highlighting}
\end{Shaded}

\begin{verbatim}
## # A tibble: 2 x 4
##     row   col expected               actual
##   <int> <int> <chr>                  <chr> 
## 1     3    NA no trailing characters abc   
## 2     4    NA no trailing characters 123.45
\end{verbatim}

\hypertarget{types-of-parse}{%
\section{\texorpdfstring{\textbf{Types of
Parse()}}{Types of Parse()}}\label{types-of-parse}}

\begin{enumerate}
\def\labelenumi{\arabic{enumi}.}
\item
  \textbf{parse\_logical()} and \textbf{parse\_integer()} parse logicals
  and integers respectively. There's basically nothing that can go wrong
  with these parsers.
\item
  \textbf{parse\_double()} is a strict numeric parser, and
  \textbf{parse\_number()} is a flexible numeric parser. These are more
  complicated than you might expect because different parts of the world
  write numbers in different ways.
\item
  \textbf{parse\_character()} seems so simple that it shouldn't be
  necessary. But one complication makes it quite important: character
  encoding.
\item
  \textbf{parse\_factor()} create factors, the data structure that R
  uses to represent categorical variables with fixed and known values.
\item
  \textbf{parse\_datetime()}, \textbf{parse\_date()}, and
  \textbf{parse\_time()} allow you to parse various date \& time
  specifications. These are the most complicated because there are so
  many different ways of writing dates.
\end{enumerate}

\hypertarget{numbers}{%
\section{\texorpdfstring{\textbf{Numbers}}{Numbers}}\label{numbers}}

It seems like it should be straightforward to parse a number, but three
problems make it tricky:

\begin{enumerate}
\def\labelenumi{\arabic{enumi}.}
\item
  People write numbers differently in different parts of the world
\item
  Numbers are often surrounded by other characters that provide some
  context, like \$1000 or 10\%
\item
  Numbers often contains ``grouping'' characters to make them easier to
  read, like ``1,000,000'', and these grouping character vary around the
  world.
\end{enumerate}

To address these problems we can use the notion of a ``locale'', an
object that specifies parsing options that differ from place to place

\begin{Shaded}
\begin{Highlighting}[]
\FunctionTok{parse\_double}\NormalTok{(}\StringTok{"1.23"}\NormalTok{)}
\end{Highlighting}
\end{Shaded}

\begin{verbatim}
## [1] 1.23
\end{verbatim}

\begin{Shaded}
\begin{Highlighting}[]
\FunctionTok{parse\_double}\NormalTok{(}\StringTok{"1,23"}\NormalTok{, }\AttributeTok{locale =} \FunctionTok{locale}\NormalTok{(}\AttributeTok{decimal\_mark =} \StringTok{","}\NormalTok{))}
\end{Highlighting}
\end{Shaded}

\begin{verbatim}
## [1] 1.23
\end{verbatim}

\begin{Shaded}
\begin{Highlighting}[]
\FunctionTok{parse\_number}\NormalTok{(}\StringTok{"$100"}\NormalTok{)}
\end{Highlighting}
\end{Shaded}

\begin{verbatim}
## [1] 100
\end{verbatim}

\begin{Shaded}
\begin{Highlighting}[]
\FunctionTok{parse\_number}\NormalTok{(}\StringTok{"20\%"}\NormalTok{)}
\end{Highlighting}
\end{Shaded}

\begin{verbatim}
## [1] 20
\end{verbatim}

\begin{Shaded}
\begin{Highlighting}[]
\FunctionTok{parse\_number}\NormalTok{(}\StringTok{"It cose $123.45"}\NormalTok{)}
\end{Highlighting}
\end{Shaded}

\begin{verbatim}
## [1] 123.45
\end{verbatim}

The final problem is addressed by the combination of
\textbf{parse\_number()} and the locale as \textbf{parse\_number()} will
ignore the ``grouping mark''

\begin{Shaded}
\begin{Highlighting}[]
\FunctionTok{parse\_number}\NormalTok{(}\StringTok{"$123,456,789"}\NormalTok{, }\AttributeTok{locale =} \FunctionTok{locale}\NormalTok{(}\AttributeTok{grouping\_mark =} \StringTok{","}\NormalTok{))}
\end{Highlighting}
\end{Shaded}

\begin{verbatim}
## [1] 123456789
\end{verbatim}

\hypertarget{strings}{%
\section{\texorpdfstring{\textbf{Strings}}{Strings}}\label{strings}}

Similar to numbers, you can use \textbf{locale} to turn character
strings into other languages

\begin{Shaded}
\begin{Highlighting}[]
\NormalTok{x2 }\OtherTok{\textless{}{-}} \StringTok{"}\SpecialCharTok{\textbackslash{}x82\textbackslash{}xb1\textbackslash{}x82\textbackslash{}xf1\textbackslash{}x82\textbackslash{}xc9\textbackslash{}x82\textbackslash{}xbf\textbackslash{}x82\textbackslash{}xcd}\StringTok{"}
\FunctionTok{parse\_character}\NormalTok{(x2, }\AttributeTok{locale =} \FunctionTok{locale}\NormalTok{(}\AttributeTok{encoding =} \StringTok{"Shift{-}JIS"}\NormalTok{))}
\end{Highlighting}
\end{Shaded}

\begin{verbatim}
## [1] "こんにちは"
\end{verbatim}

\hypertarget{factors}{%
\section{\texorpdfstring{\textbf{Factors}}{Factors}}\label{factors}}

Within the \textbf{parse\_factor()} function, we can use
\textbf{levels=} to signify how exactly to sort the strings as factors.

\begin{Shaded}
\begin{Highlighting}[]
\NormalTok{fruit }\OtherTok{\textless{}{-}} \FunctionTok{c}\NormalTok{(}\StringTok{"apple"}\NormalTok{, }\StringTok{"banana"}\NormalTok{)}
\FunctionTok{parse\_factor}\NormalTok{(}\FunctionTok{c}\NormalTok{(}\StringTok{"apple"}\NormalTok{, }\StringTok{"banana"}\NormalTok{, }\StringTok{"banana"}\NormalTok{), }\AttributeTok{levels =}\NormalTok{ fruit)}
\end{Highlighting}
\end{Shaded}

\begin{verbatim}
## [1] apple  banana banana
## Levels: apple banana
\end{verbatim}

\hypertarget{dates-date-times-and-times}{%
\section{\texorpdfstring{\textbf{Dates, date-times, and
times}}{Dates, date-times, and times}}\label{dates-date-times-and-times}}

You pick between three parsers depending on whether you want a date (the
number of days since 1970-01-01), a date-time (the number of seconds
since mignight 1970-01-01), or a time (the number of seconds since
midnight). When called without any additional arguments.

\begin{Shaded}
\begin{Highlighting}[]
\FunctionTok{parse\_datetime}\NormalTok{(}\StringTok{"2010{-}10{-}01T2010"}\NormalTok{)}
\end{Highlighting}
\end{Shaded}

\begin{verbatim}
## [1] "2010-10-01 20:10:00 UTC"
\end{verbatim}

\textbf{parse\_date()} expects a four digit year, a \textbf{-} or
\textbf{/}, the month, a \textbf{-} or \textbf{/}, then the day:

\begin{Shaded}
\begin{Highlighting}[]
\FunctionTok{parse\_date}\NormalTok{(}\StringTok{"2010{-}10{-}01"}\NormalTok{)}
\end{Highlighting}
\end{Shaded}

\begin{verbatim}
## [1] "2010-10-01"
\end{verbatim}

\textbf{parse\_time()} expects the hour, \textbf{:}, minutes, optionally
\textbf{:} and seconds, and an optional am/pm specifer.

\begin{Shaded}
\begin{Highlighting}[]
\FunctionTok{parse\_time}\NormalTok{(}\StringTok{"01:10am"}\NormalTok{)}
\end{Highlighting}
\end{Shaded}

\begin{verbatim}
## 01:10:00
\end{verbatim}

\begin{Shaded}
\begin{Highlighting}[]
\FunctionTok{parse\_time}\NormalTok{(}\StringTok{"20:10:01"}\NormalTok{)}
\end{Highlighting}
\end{Shaded}

\begin{verbatim}
## 20:10:01
\end{verbatim}

If these defaults don't work for your data you can supply your own
date-time format, built up of the following pieces: \emph{Year}

\textbf{\%Y} (4digits) \textbf{\%y} (2 digits)

\emph{Month} \textbf{\%m} (2 digits) \textbf{\%b} (abbreviated name)
\textbf{B\%} (full name, ``January'')

\emph{Day} \textbf{\%d} (2 digits) \textbf{\%e} (optional leading space)

\emph{Time} \textbf{\%H} (0-23 hour) \textbf{\%I} (0-12, must be used
with \%p) \textbf{\%p} (AM/PM indicator) \textbf{\%M} (minutes)
\textbf{\%S} (integer seconds) \textbf{\%OS} (Real seconds) \textbf{\%Z}
(Time Zone) \textbf{\%z} (as offset from UTC, eg +0800)

Examples

\begin{Shaded}
\begin{Highlighting}[]
\FunctionTok{parse\_date}\NormalTok{(}\StringTok{"01/02/15"}\NormalTok{, }\StringTok{"\%m/\%d/\%y"}\NormalTok{)}
\end{Highlighting}
\end{Shaded}

\begin{verbatim}
## [1] "2015-01-02"
\end{verbatim}

\begin{Shaded}
\begin{Highlighting}[]
\FunctionTok{parse\_date}\NormalTok{(}\StringTok{"01/02/15"}\NormalTok{, }\StringTok{"\%d/\%m/\%y"}\NormalTok{)}
\end{Highlighting}
\end{Shaded}

\begin{verbatim}
## [1] "2015-02-01"
\end{verbatim}

\hypertarget{parsing-a-file}{%
\section{\texorpdfstring{\textbf{Parsing a
File}}{Parsing a File}}\label{parsing-a-file}}

readr uses a heuristic to figure out the type of each column: it reads
the first 1000 rows and uses some (moderately conservative) heuristics
to figure out the type of each column. You can emulate this process with
a character vector using \textbf{guess\_parser()}, which returns readr's
best guess, and \textbf{parse\_guess()} which uses that guess to parse a
column.

\begin{Shaded}
\begin{Highlighting}[]
\FunctionTok{guess\_parser}\NormalTok{(}\StringTok{"2010{-}10{-}01"}\NormalTok{)}
\end{Highlighting}
\end{Shaded}

\begin{verbatim}
## [1] "date"
\end{verbatim}

\begin{Shaded}
\begin{Highlighting}[]
\FunctionTok{guess\_parser}\NormalTok{(}\FunctionTok{c}\NormalTok{(}\StringTok{"TRUE"}\NormalTok{, }\StringTok{"FALSE"}\NormalTok{, }\StringTok{"T"}\NormalTok{))}
\end{Highlighting}
\end{Shaded}

\begin{verbatim}
## [1] "logical"
\end{verbatim}

\begin{Shaded}
\begin{Highlighting}[]
\FunctionTok{str}\NormalTok{(}\FunctionTok{parse\_guess}\NormalTok{(}\StringTok{"2010{-}10{-}01"}\NormalTok{))}
\end{Highlighting}
\end{Shaded}

\begin{verbatim}
##  Date[1:1], format: "2010-10-01"
\end{verbatim}

\hypertarget{writing-to-a-file}{%
\section{\texorpdfstring{\textbf{Writing to a
File}}{Writing to a File}}\label{writing-to-a-file}}

readr also comes with two useful functions for writing data back to
disk: \textbf{write\_csv()} and \textbf{write\_tsv()}.

\begin{Shaded}
\begin{Highlighting}[]
\NormalTok{challenge }\OtherTok{\textless{}{-}} \FunctionTok{read\_csv}\NormalTok{(}\FunctionTok{readr\_example}\NormalTok{(}\StringTok{"challenge.csv"}\NormalTok{))}
\end{Highlighting}
\end{Shaded}

\begin{verbatim}
## Rows: 2000 Columns: 2
## -- Column specification --------------------------------------------------------
## Delimiter: ","
## dbl  (1): x
## date (1): y
## 
## i Use `spec()` to retrieve the full column specification for this data.
## i Specify the column types or set `show_col_types = FALSE` to quiet this message.
\end{verbatim}

\begin{Shaded}
\begin{Highlighting}[]
\FunctionTok{write\_csv}\NormalTok{(challenge, }\StringTok{"challenge.csv"}\NormalTok{)}
\end{Highlighting}
\end{Shaded}


\end{document}
